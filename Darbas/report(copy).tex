\documentclass[conference]{IEEEtran}
\IEEEoverridecommandlockouts

\usepackage[utf8]{inputenc}
\usepackage[lithuanian]{babel}
\usepackage[T1]{fontenc}


% The preceding line is only needed to identify funding in the first footnote. If that is unneeded, please comment it out.
\usepackage{cite}
\usepackage{amsmath,amssymb,amsfonts}
\usepackage{algorithmic}
\usepackage{graphicx}
\usepackage{textcomp}
\usepackage{xcolor}
\def\BibTeX{{\rm B\kern-.05em{\sc i\kern-.025em b}\kern-.08em
    T\kern-.1667em\lower.7ex\hbox{E}\kern-.125emX}}

\usepackage[labelsep=endash]{caption}
\renewcommand{\figurename}{pav}


\usepackage{lipsum}  




\begin{document}

\title{Komandinės užduoties tema}

\author{\IEEEauthorblockN{Vardas Pavardė}
\IEEEauthorblockA{\textit{Informatikos institutas} \\
\textit{Matematikos if informatikos fakultetas}\\
Vilnius, Lietuva \\
epastas@epastas.lt}
\and
\IEEEauthorblockN{Vardas Pavardė}
\IEEEauthorblockA{\textit{Informatikos institutas} \\
\textit{Matematikos if informatikos fakultetas}\\
Vilnius, Lietuva \\
epastas@epastas.lt}
\and
\IEEEauthorblockN{Vardas Pavardė}
\IEEEauthorblockA{\textit{Informatikos institutas} \\
\textit{Matematikos if informatikos fakultetas}\\
Vilnius, Lietuva \\
epastas@epastas.lt}
\and
\IEEEauthorblockN{Vardas Pavardė}
\IEEEauthorblockA{\textit{Informatikos institutas} \\
\textit{Matematikos if informatikos fakultetas}\\
Vilnius, Lietuva \\
epastas@epastas.lt}
\and
\IEEEauthorblockN{Vardas Pavardė}
\IEEEauthorblockA{\textit{Informatikos institutas} \\
\textit{Matematikos if informatikos fakultetas}\\
Vilnius, Lietuva \\
epastas@epastas.lt}
\and
\IEEEauthorblockN{Vardas Pavardė}
\IEEEauthorblockA{\textit{Informatikos institutas} \\
\textit{Matematikos if informatikos fakultetas}\\
Vilnius, Lietuva \\
epastas@epastas.lt}
}

\maketitle

\begin{abstract}
Santrauka kokia problema spręsta. Ką taikėte, kokie rezultatai. 
\end{abstract}

\begin{IEEEkeywords}
5-6 Raktiniai žodžiai
\end{IEEEkeywords}

\section{Įvadas}
Pristatote problemą

\section{Metodai}

\subsection{Metodas A}

Pristatote naudojamus metodus

\begin{equation}
\mathcal{L}({\bf X}, {\bf Y}) = \frac{1}{w \cdot h} \sum_{i=1}^h \sum_{j=1}^w (X_{i, j} - Y_{i, j})^2
\label{eq:lygtis1}
\end{equation}

Taikyta nuostolių funkcija ~\eqref{eq:lygtis1}.

\begin{align}
y & = f(x) \nonumber \\
f & = f_1(f_2(x))
\label{eq:lygtis2}
\end{align}

Taikytas modelis ~\eqref{eq:lygtis2}.


\section{Duomenys}
Aprašote naudojamus duomenis.

\subsection{Equations}



\subsection{Paveikslėliai ir lentelės}

Paveikslėlį cituojame ``~\ref{fig} pav.''.

Paveikslėlį cituojame ``~\ref{tab1} lentelė''.

\begin{table}[htbp]
\caption{Lentelės aprašas}
\begin{center}
\begin{tabular}{|c|c|c|c|}
\hline
a & b & c &  d \\
\hline
\end{tabular}
\label{tab1}
\end{center}
\end{table}

\begin{figure}[!h] % įterpti čia
\centerline{\includegraphics{fig1.png}}
\caption{Paveikslėlio aprašas.}
\label{fig}
\end{figure}


\section{Rezultatai}

\begin{figure*}[ht] % puslapio viršuje
\centerline{\includegraphics{fig1.png}}
\caption{Paveikslėlio aprašas.}
\label{fig}
\end{figure*}


\lipsum[2-10]






Cituojame šaltinį \cite{lecun2015deep}.

\bibliographystyle{plain}
\bibliography{saltiniai}

\end{document}
